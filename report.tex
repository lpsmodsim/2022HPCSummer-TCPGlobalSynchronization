\documentclass{article}

%graphics
\usepackage{graphicx}
\graphicspath{{./images/}}

\usepackage{float}

% margins of 1 inch:
\setlength{\topmargin}{-.5in}
\setlength{\textheight}{9.5in}
\setlength{\oddsidemargin}{0in}
\setlength{\textwidth}{6.5in}

\usepackage{hyperref}
\hypersetup{
    colorlinks=true,
    linkcolor=blue,
    filecolor=magenta,      
    urlcolor=cyan,
    pdftitle={Overleaf Example},
    pdfpagemode=FullScreen,
    }

\begin{document}

    % https://stackoverflow.com/a/3408428/1164295
    \begin{minipage}[h]{\textwidth}
        \title{2022 Future Computing Summer Internship Project:\\(Mapping TCP Global Synchronization to the SST model)}
        \author{Nicholas Schantz\footnote{nickjohnschantz@gmail.com}\ , 
        anotherfirst\footnote{anemail@domain.com}}
        \date{\today}
            \maketitle
        \begin{abstract}
            TCP Global Synchronization is a networking problem in which a burst of traffic in a network causes multiple clients to drop packets and limit their transmission rates. The clients begin to increase their transmission rates consecutively leading to more packet loss and transmission limiting, creating a loop of this activity. SST is used to model this activity and determine metrics that can determine if this problem exist in a simulation. A key metric found is to look in a window of activity when packet loss occurs and count how many clients have reduced their transmission rates.
        \end{abstract}
    \end{minipage}

\ \\
% see https://en.wikipedia.org/wiki/George_H._Heilmeier#Heilmeier's_Catechism

%\maketitle

\section{Project Description} % what problem is being addressed? 

The challenge addressed by this work is to map the networking problem 'TCP Global Synchronization' to the SST model. The problem is studied to understand the mathematic conditions that create this problem. This information is used to create a SST model and simulate it to identify metrics to detect that the problem has occurred in simulated systems.

\section{Motivation} % Why does this work matter? Who cares? If you're successful, what difference does it make?

Identifying the metrics for detecting TCP Global Synchronization will be vital for developing distributed systems that can avoid congestion collapse from occurring during communication. Furthermore, the metrics answer the question as to why the problem has occurred in the system. To add, the SST models written are resources that other users can use to learn and utilize SST's discrete event simulator.

\section{Prior work} % what does this build on?



\section{How to do the thing}

The software developed to respond to this challenge was run on one laptop.
The software is available on (https://github.com/lpsmodsim/2022HPCSummer-TCPGlobalSynchronization)

\section{Result} % conclusion/summary

Potential metrics included the following:\newline
	$\bullet$ Average aggregate link utilization \newline
	$\bullet$ Queue size fluctuation \newline
	$\bullet$ Time in which clients drop \newline\newline

	Average aggregate link utilization was chosen as a unique pattern forms in the link utilization when global synchronization happens. Due to clients syncing and lowering transmission rates all at once, the link can rest for a bit and clear up its queue and will process messages under its capacity, lowering link utilization. As transmission rates increase, link utilization will grow to 100\% and this cycle will repeat when packet loss occurs again.\newline
	
	\begin{figure}[H]
	\caption{SST Model collected data of aggregate link utilization over 800 seconds.}
	\centering
	\includegraphics{linkutil}
	\end{figure}
	
	However, this metric was thrown out when I researched around RED which is a dropping mechanism in reliable networks that attempts to avoid global synchronization. Average aggregate link utilization under this dropping mechanism appears to be similar to tail drop except the average is closer to peak utilization. In this case using this metric may alert on false positives. \newline
	
	The queue size fluctuates quite often during global synchronization as it will empty out more often during lower link utilization while it will grow during peak link utilization. I thought it would be possible to measure in a window the time span in which the queue size is emptied out and growing to max capacity and seeing if that span of time matches the duration between two transmission rate limits of a client.
	(Potential picture)
	However, during periods of low latency and high link utilization, the queue will be fluctuating drastically. (Explain). Therefore this metric was tossed out.

\begin{thebibliography}{9}
\bibitem{texbook}
ADD BIB FILE

\bibitem{lamport94}
ADD BIB FILE
\end{thebibliography}

\end{document}
